% IMPORTANTE
% Para poner barrabaja en una url hay que cambiar '_' por '\textunderscore'

%========================================================================================
% PAQUETES
%========================================================================================

\usepackage[spanish]{babel} % Indicar idioma español
\usepackage[utf8]{inputenc} % Poder poner tildes directamente
\usepackage{graphicx} % Para incluir gráficos
\usepackage{cite} % Para crear referencias
\usepackage{svg} % Para usar imágenes en formato vectorial
\usepackage{float} % Para modificar como aparecen las imagenes
\usepackage{listings} % Para introducir código
\usepackage{times}
\usepackage{color} % para añadir colores
\usepackage{hyperref}  % Para que se creen enlaces en el documento (a otras partes o a la web)
\usepackage{AnonymousPro}
\usepackage{url}
\usepackage[Bjornstrup]{fncychap} % Para poner los títulos de los capítulos más bonitos
\usepackage{fancyhdr} % Para configurar las cabeceras y los piés de página
\usepackage{titling} % Para usar la información del documento en otras partes de él

%========================================================================================
% ESTILO
%========================================================================================

%----------------------------------------------------------------------------------------
% RUTAS PARA LAS IMAGENES
%----------------------------------------------------------------------------------------

\graphicspath{{./figuras/}}

%----------------------------------------------------------------------------------------
% TIPOGRAFIA
%----------------------------------------------------------------------------------------

\renewcommand{\familydefault}{\sfdefault} % Cambia la tipografía de todo el documento
\setlength{\parskip}{2mm} % Define la distancia entre parrafos (por defecto = 0)

%----------------------------------------------------------------------------------------
% CABECERAS Y PIES DE PAGINA
%----------------------------------------------------------------------------------------
\pagestyle{fancy}
\fancyhf{}
\fancyhead[CE,CO]{\leftmark}
\fancyhead[LE,RO]{\thepage}
\fancyfoot[RE,LO]{Julen Aristimuño, Ander Granado, Joseba Ruiz}
\fancyfoot[LE,RO]{\thepage}

\renewcommand{\headrulewidth}{2pt}
\renewcommand{\footrulewidth}{1pt}

%----------------------------------------------------------------------------------------
% ENLACES
%----------------------------------------------------------------------------------------

\hypersetup{
	colorlinks,
	citecolor=black,
	filecolor=black,
	linkcolor=black,
	urlcolor=black
}

%----------------------------------------------------------------------------------------
% NUMERO DE PROFUNDIDAD DE NUMERACION DE CONENIDOS
%----------------------------------------------------------------------------------------
\setcounter{tocdepth}{2}
\setcounter{secnumdepth}{2}

%----------------------------------------------------------------------------------------
% CODIGO
%----------------------------------------------------------------------------------------

\definecolor{gray95}{gray}{.95}
\definecolor{gray85}{gray}{.80}
\definecolor{gray45}{gray}{.45}
\definecolor{myturquoise}{RGB}{0, 128, 128}

\lstset{ 
	frame=Ltb,
	framerule=0pt,
	aboveskip=0.5cm,
	framextopmargin=3pt,
	framexbottommargin=3pt,
	framexleftmargin=0.2cm,
	framesep=0pt,
	rulesep=2.0pt,
	backgroundcolor=\color{gray95},
	rulesepcolor=\color{black},
	%
	stringstyle=\ttfamily,
	showstringspaces = false,
	basicstyle=\small\ttfamily,
	commentstyle=\color{gray45},
	keywordstyle=\bfseries,
	%
	numbers=left,
	numbersep=15pt,
	numberstyle=\tiny,
	numberfirstline = false,
	breaklines=true,
}

% Minimizar fragmentado de listados
\lstnewenvironment{listing}[1][]
{\lstset{#1}\pagebreak[0]}{\pagebreak[0]}

% Estilo de codigo para la consola
\lstdefinestyle{consola}{
	language=bash,
	breaklines=true,
	basicstyle=\footnotesize\bf\ttfamily,
	backgroundcolor=\color{gray85},
	rulesepcolor=\color{myturquoise},
	numbers=none,
}