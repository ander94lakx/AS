\chapter{Prueba con nodos ROS}
Vamos a crear un pequeño ejemplo de este tipo para comprender mejor su funcionamiento, pero con todos los nodos en el host, para asimilar el proceso de crear y probar paquetes con ROS. En el ejemplo, El publisher publicará una serie de datos, que será recibidos por el subscriber una vez se haya suscrito al topic en el que se estén publicando. Para hacerlo lo más simple posible, los datos que se pasarán serán cadenas de carácteres.

Nos basaremos en uno de los ejemplos de los que disponemos en la Wiki de ROS \cite{ros-tutorials}, al que haremos unos pequeños cambios para adaptarlo.

	\section{Código de la prueba}
	El código se divide en dos ficheros .cpp en los que tendremos por una parte el publisher y en la otra el subscriber.	Tendremos un tercer ahrivo más, que será el archivo CMakeLists.txt que nos servirá a la hora de construir el paquete.
	
		\subsection{Publisher}
		\lstinputlisting[style=cpp,numbers=left]{../code/pub-sub-example/talker.cpp}
		
		\subsection{Subscriber}
		\lstinputlisting[style=cpp,numbers=left]{../code/pub-sub-example/listener.cpp}
		
		\subsection{CMakeLists}
		\lstinputlisting[style=makefile,numbers=left]{../code/pub-sub-example/CMakeLists.txt}
	
	\section{Construir el paquete}
	Una vez programado el sistema, pasamos a compilarlo y construirlo. Basándonos en lo que se comentó en el capítulo anterior, lo haremos de la siguiente manera.
	
	\begin{lstlisting}[style=consola,numbers=left]
~$ mkdir -p ~/catkin_ws/src
~$ cd ~/catkin_ws/src
~/catkin_ws/src$ catkin_init_workspace
Creating symlink "/home/ander/catkin_ws/src/CMakeLists.txt" pointing to "/opt/ros/indigo/share/catkin/cmake/toplevel.cmake"
~/catkin_ws/src$ cd ..
~/catkin_ws$ source devel/setup.bash
bash: devel/setup.bash: No existe el archivo o el directorio
~/catkin_ws$ cd src/
~/catkin_ws/src$ catkin_create_pkg prueba std_msgs rospy roscpp
Created file prueba/package.xml
Created file prueba/CMakeLists.txt
Created folder prueba/include/prueba
Created folder prueba/src
Successfully created files in /home/ander/catkin_ws/src/prueba. Please adjust the values in package.xml.
~/catkin_ws/src$ ls
CMakeLists.txt  prueba
~/catkin_ws/src$ cd ..
~/catkin_ws$ catkin_make
Base path: /home/ander/catkin_ws
Source space: /home/ander/catkin_ws/src
Build space: /home/ander/catkin_ws/build
Devel space: /home/ander/catkin_ws/devel
Install space: /home/ander/catkin_ws/install
####
#### Running command: "cmake /home/ander/catkin_ws/src -DCATKIN_DEVEL_PREFIX=/home/ander/catkin_ws/devel -DCMAKE_INSTALL_PREFIX=/home/ander/catkin_ws/install -G Unix Makefiles" in "/home/ander/catkin_ws/build"
####
-- The C compiler identification is GNU 4.8.4
-- The CXX compiler identification is GNU 4.8.4
-- Check for working C compiler: /usr/bin/cc
-- Check for working C compiler: /usr/bin/cc -- works
-- Detecting C compiler ABI info
-- Detecting C compiler ABI info - done
-- Check for working CXX compiler: /usr/bin/c++
-- Check for working CXX compiler: /usr/bin/c++ -- works
-- Detecting CXX compiler ABI info
-- Detecting CXX compiler ABI info - done
-- Using CATKIN_DEVEL_PREFIX: /home/ander/catkin_ws/devel
-- Using CMAKE_PREFIX_PATH: /opt/ros/indigo
-- This workspace overlays: /opt/ros/indigo
-- Found PythonInterp: /usr/bin/python (found version "2.7.6")
-- Using PYTHON_EXECUTABLE: /usr/bin/python
-- Using Debian Python package layout
-- Using empy: /usr/bin/empy
-- Using CATKIN_ENABLE_TESTING: ON
-- Call enable_testing()
-- Using CATKIN_TEST_RESULTS_DIR: /home/ander/catkin_ws/build/test_results
-- Looking for include file pthread.h
-- Looking for include file pthread.h - found
-- Looking for pthread_create
-- Looking for pthread_create - not found
-- Looking for pthread_create in pthreads
-- Looking for pthread_create in pthreads - not found
-- Looking for pthread_create in pthread
-- Looking for pthread_create in pthread - found
-- Found Threads: TRUE  
-- Found gtest sources under '/usr/src/gtest': gtests will be built
-- Using Python nosetests: /usr/bin/nosetests-2.7
-- catkin 0.6.14
-- BUILD_SHARED_LIBS is on
-- ~~~~~~~~~~~~~~~~~~~~~~~~~~~~~~~~~~~~~~~~~~~~~~~~~
-- ~~  traversing 1 packages in topological order:
-- ~~  - prueba
-- ~~~~~~~~~~~~~~~~~~~~~~~~~~~~~~~~~~~~~~~~~~~~~~~~~
-- +++ processing catkin package: 'pruebaROS'
-- ==> add_subdirectory(pruebaROS)
-- Configuring done
-- Generating done
-- Build files have been written to: /home/ander/catkin_ws/build
####
#### Running command: "make -j1 -l1" in "/home/ander/catkin_ws/build"
####
~/catkin_ws$ ls
build  devel  src
~/catkin_ws$ cd src
~/catkin_ws/src$ ls
CMakeLists.txt  prueba
~/catkin_ws/src$ cd prueba/
~/catkin_ws/src/prueba$ ls
CMakeLists.txt  include  package.xml  src
~/catkin_ws/src/prueba$ cd src
~/catkin_ws/src/prueba/src$ ls
~/catkin_ws/src/prueba/src$ touch talker.cpp
~/catkin_ws/src/prueba/src$ touch listener.cpp
~/catkin_ws/src/prueba/src$ ls
listener.cpp  talker.cpp

	\end{lstlisting}

	Ahora ya tenemos todos los archivos creados, debemos llenarlos con el contenido que hemos mostrado antes. En este caso uso \textit{nano} para editar los archivos, pero se puede hacer con \textit{vi} o cualquier otro editor.

	\begin{lstlisting}[style=consola,numbers=left]
~/catkin_ws/src/prueba/src$ nano listener.cpp
~/catkin_ws/src/prueba/src$ nano talker.cpp
~/catkin_ws/src/prueba/src$ cd ..
~/catkin_ws/src/prueba$ ls
CMakeLists.txt  include  package.xml  src
~/catkin_ws/src/prueba$ echo "" > CMakeLists.txt # Para vaciar el archivo
~/catkin_ws/src/prueba$ nano CMakeLists.txt
	\end{lstlisting}

	Una vez llenos podemos proceder a compilarlo.
	
	\begin{lstlisting}[style=consola,numbers=left]
~/catkin_ws/src/prueba$ cd ..
~/catkin_ws/src$ cd ..
~/catkin_ws$ ls
build  devel  src
~/catkin_ws$ catkin_make
ander@ubuntu-VirtualBox:~/catkin_ws$ catkin_make
Base path: /home/ander/catkin_ws
Source space: /home/ander/catkin_ws/src
Build space: /home/ander/catkin_ws/build
Devel space: /home/ander/catkin_ws/devel
Install space: /home/ander/catkin_ws/install
####
#### Running command: "cmake /home/ander/catkin_ws/src -DCATKIN_DEVEL_PREFIX=/home/ander/catkin_ws/devel -DCMAKE_INSTALL_PREFIX=/home/ander/catkin_ws/install -G Unix Makefiles" in "/home/ander/catkin_ws/build"
####
-- Using CATKIN_DEVEL_PREFIX: /home/ander/catkin_ws/devel
-- Using CMAKE_PREFIX_PATH: /home/ander/catkin_ws/devel;/opt/ros/indigo
-- This workspace overlays: /home/ander/catkin_ws/devel;/opt/ros/indigo
-- Using PYTHON_EXECUTABLE: /usr/bin/python
-- Using Debian Python package layout
-- Using empy: /usr/bin/empy
-- Using CATKIN_ENABLE_TESTING: ON
-- Call enable_testing()
-- Using CATKIN_TEST_RESULTS_DIR: /home/ander/catkin_ws/build/test_results
-- Found gtest sources under '/usr/src/gtest': gtests will be built
-- Using Python nosetests: /usr/bin/nosetests-2.7
-- catkin 0.6.14
-- BUILD_SHARED_LIBS is on
-- ~~~~~~~~~~~~~~~~~~~~~~~~~~~~~~~~~~~~~~~~~~~~~~~~~
-- ~~  traversing 1 packages in topological order:
-- ~~  - prueba
-- ~~~~~~~~~~~~~~~~~~~~~~~~~~~~~~~~~~~~~~~~~~~~~~~~~
-- +++ processing catkin package: 'prueba'
-- ==> add_subdirectory(prueba)
-- Configuring done
-- Generating done
-- Build files have been written to: /home/ander/catkin_ws/build
####
#### Running command: "make -j1 -l1" in "/home/ander/catkin_ws/build"
####
Scanning dependencies of target listener
[ 50%] Building CXX object prueba/CMakeFiles/listener.dir/src/listener.cpp.o
Linking CXX executable /home/ander/catkin_ws/devel/lib/prueba/listener
[ 50%] Built target listener
Scanning dependencies of target talker
[100%] Building CXX object prueba/CMakeFiles/talker.dir/src/talker.cpp.o
Linking CXX executable /home/ander/catkin_ws/devel/lib/prueba/talker
[100%] Built target talker
	\end{lstlisting}
	
	Esto nos generará dos ejecutables, \emph{talker} y \emph{listener}, que por defecto irán al directorio devel/lib/[nombre del paquete] de nuestro workspace.
	
	\section{Ejecución}
	Probaremos el sistema para comprobar que funciona correctamente. Para ello haremos lo siguiente.
	
	\begin{enumerate}
		\item Por una parte lanzamos \emph{roscore}.
		\begin{lstlisting}[style=consola,numbers=left]
$ roscore
		\end{lstlisting}
		
		\item Compilamos de nuevo el workspace que tenemos creado.
		\begin{lstlisting}[style=consola,numbers=left]
$ cd ~/catkin_ws
$ . devel/setup.bash
$ catkin_make
		\end{lstlisting}
		
		\item Lanzamos por otra parte el \emph{listener} de la siguiente manera.
		\begin{lstlisting}[style=consola,numbers=left]
$ rosrun prueba listener

		\end{lstlisting}

		\item Mandamos un mensaje mediante el \emph{talker}.
		\begin{lstlisting}[style=consola,numbers=left]
$ rostopic pub -1 /chatter std_msgs/String PruebaMensaje
publishing and latching message for 3.0 seconds
		\end{lstlisting}
		
		\begin{lstlisting}[style=consola,numbers=left]
$ rosrun prueba listener
[ INFO] [1447683677.299227505]: I heard: [PruebaMensaje]
		\end{lstlisting}
		
		Se puede apreciar que el listener recibe el mensaje enviado por el talker.
		
	\end{enumerate}
