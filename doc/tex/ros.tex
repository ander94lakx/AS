\chapter{Introducción a ROS}
Anteriormente se ha explicado qué es ROS y que características tiene. En este capitulo se pasara a profundizar en su funcionamiento, y mostraremos como se pueden programar los sistemas de paso de mensajes que pasaremos a implementar en un sistema de contenedores Docker.

	\section{Entorno de trabajo}
	Para poder trabajar con ROS lo primero que debemos tener es un entorno con las herramientas de ROS instaladas. En este caso, como vamos a usar contenedores Docker con todo lo necesario en ellos no necesitaremos instalar ningún tipo de paquete adicional. El Dockerfile que vamos a usar para generar la imagen Ubuntu con ROS ya instalado se encuentra disponible en el Docker Hub y aparece con el nombre \emph{osrf/ros:indigo-desktop}. Esta imagen contiene \cite{ros-installation}:
	\begin{itemize}
		\item \emph{\textbf{ros-base}}, la base de ROS, que a su vez contiene:
		\begin{itemize}
			\item \emph{\textbf{ros-core}}, el núcleo de ROS
			\item Librerías para construir aplicaciones
			\item Librerías para comunicación
		\end{itemize}
		\item \emph{\textbf{rqt}}, framework basado en Qt para construir aplicaciones con interfaz gráfica de usuario (GUI) con ROS
		\item \emph{\textbf{rviz}}, herramienta de visualización 3D para ROS
		\item Librerías genéricas para sistemas robóticos
	\end{itemize}
	
	De momento no vamos a hacer uso de ninguna herramienta gráfica. Para probar el funcionamiento de ROS antes de programar sobre nuestro sistema, vamos a elaborar una pequeña aplicación distribuida que nos permita enviar unos datos de un nodo a otro.
	
	\section{Uso de ROS}
	Al igual que hacemos con Docker, manejaremos ROS desde la terminal. Existen una serie de comandos que necesitamos conocer para poder empezar a trabajar con él. Los comandos son los siguientes \cite{ros-comandos}:
	
	\begin{itemize}
		\item \emph{\textbf{roscd}}: cambia a un directorio de paquete
		\item \emph{\textbf{roscore}}: ejecuta todo lo necesario para que dar soporte de ejecución al sistema completo de ROS. Siempre tiene que estar ejecutándose para permitir que se comuniquen los nodos. Permite ejecutarse en un determinado puerto
		\item \emph{\textbf{rosnode}}: nos proporciona información sobre un nodo. Disponemos de las siguientes opciones:
		\begin{itemize}
			\item \textit{rosnode info [nodo]}: muestra información sobre el nodo
			\item \textit{rosnode kill [nodo]}: mata ese proceso
			\item \textit{rosnode list}: muestra los nodos ejecutándose
			\item \textit{rosnode machine [maquina]}: muestra los nodos que se están ejecutando en la máquina
			\item \textit{rosnode ping [nodo]}: comprueba la conectividad del nodo
		\end{itemize}
		\item \emph{\textbf{rosrun} [nombre\_paquete] [nombre\_nodo]}: permite ejecutar cualquier aplicación de un paquete sin necesidad de cambiar a su directorio
		\item \emph{\textbf{rostopic}}: permite obtener información sobre un topic
		\begin{itemize}
			\item \textit{rostopic bw [topic]}: muestra el ancho de banda consumido por un tópico
			\item \textit{rostopic echo [topic]}: imprime datos del topic por la salida estándar
			\item \textit{rostopic find}: encuentra un topic
			\item \textit{rostopic info [topic]}: imprime información de un topic
			\item \textit{rostopic list}: imprime información sobre los topics activos
			\item \textit{rostopic pub [topic]}: publica datos a un topic activo
			\item \textit{rostopic type [topic]}: imprime el tipo de información de un topic
		\end{itemize}
		\item \emph{\textbf{roswtf}}: permite comprobar si algo va mal
	\end{itemize}
	
	\section{Gestionar paquetes ROS}
	Cualquier aplicación que hagamos con ROS será un paquete que podremos instalar. ROS dispone de varias herramientas mediante las cuales crear y compilar esos paquetes. la más famosa de estas herramientas es \emph{\textbf{catkin}}. \emph{catkin} permite crear y compilar paquetes (que se denominan paquetes \emph{catkin}). que contendrán todo el código que desarrollemos. Se basa en el funcionamiento de herramientas como Make o CMake, habituales para compilar aplicaciones en entornos linux.
	
	Para que un paquete se pueda considerar un paquete \emph{catkin} tiene que cumplir una serie de requisitos \cite{ros-catkin-create}:
	
	\begin{enumerate}
		\item El paquete debe contener un fichero de llamado \textit{package.xml} con un formato concreto.
		\begin{itemize}
			\item Este archivo contendrá diferentes metadatos sobre el paquete
		\end{itemize}
		\item El paquete debe contener un fichero \textit{CMakeLists.txt} que usará catkin a la hora de compilar el paquete
		\item No puede haber más de un paquete por carpeta
		\begin{itemize}
			\item No puede haber paquetes anidados en otros paquetes ni varios paquetes en una misma carpeta
		\end{itemize}
	\end{enumerate}
	
		\subsection{Crear paquetes}
		Para crear un paquete con catkin primero debemos crear un workspace de catkin, que es quien contendrá los paquetes que creemos. Lo haremos de la siguiente manera \cite{ros-catkin-workspace}.
		
		\begin{lstlisting}[style=consola,numbers=left]
$ mkdir -p ~/catkin_ws/src
$ cd ~/catkin_ws/src
$ catkin_init_workspace
# Aunque no haya nada compilamos el workspace
$ cd ~/catkin_ws/
$ source devel/setup.bash
		\end{lstlisting}
		
		Una vez creado el workspace podemos crear un paquete de la siguiente manera.
		
		\begin{lstlisting}[style=consola,numbers=left]
$ cd ~/catkin_ws/src
$ catkin_create_pkg paquete std_msgs rospy roscpp
		\end{lstlisting}
		
		Como se puede observar, hemos creado un paquete llamado \textit{paquete}. Los siguientes nombres que le hemos indicado son las dependencia que tiene el paquete, en este caso son dependencias para compilar código fuente de C++ y Python, además de herramientas para manejar mensajes.
		
		\subsection{Compilar paquetes}
		
		Para compilar paquetes generados mediante catkin, se hace uso del comando \emph{catkin-make}. Este comando compila todo en función el archivo \emph{CMakeLists.txt} para saber como debe compilar el paquete.
		
		\begin{lstlisting}[style=consola,numbers=left]
$ cd ~/catkin_ws/
$ catkin_make
		\end{lstlisting}
		
	\section{Modelo distribuido Publisher-Subscriber}
	
	En redes existe un tipo de comunicación conocida como modelo \emph{Publisher-Subscriber}. Esta metodología consiste en dos nodos, uno conocido como \emph{publisher} que va publicando mensajes. Éste nodo no decide a quién se mandan los mensajes, sino que el otro nodo, el nodo \emph{subscriber}, tal y como su nombre indica, se suscribe al nodo que publica los mensajes, y a partir de ese momento recibe los mensajes que va publicando.
	
	En realidad, para separar diferentes tipos de mensajes, se hace uso de \emph{topics} (temas). Esto permite que un publisher pueda tener varios canales de envío de mensajes. Debido a esto, lo que realmente hacen los subscribers es suscribirse a esos topics, y no a los publishers. A su vez, Los publishers deciden en que topic publicar los mensajes.
	
	\begin{figure}[H]
		\centering
		\includegraphics[width=1.0\textwidth]{ros-pub-sub}
		\caption{Modelo Publisher-Subscriber}
		\label{fig:ros-pub-sub}
	\end{figure}