% Estilo
\renewcommand{\familydefault}{\sfdefault} % Cambia la tipografía de todo el documento
\setlength{\parskip}{2mm} % Define la distancia entre parrafos (por defecto = 0)

% Estilo de los enlaces
\hypersetup{
	colorlinks,
	citecolor=black,
	filecolor=black,
	linkcolor=black,
	urlcolor=black
}

% -- ESTILO PARA EL CODIGO -------------------------------------------
\definecolor{gray95}{gray}{.95}
\definecolor{gray85}{gray}{.80}
\definecolor{gray45}{gray}{.45}
\definecolor{myturquoise}{RGB}{0, 128, 128}

\lstset{ 
	frame=Ltb,
	framerule=0pt,
	aboveskip=0.5cm,
	framextopmargin=3pt,
	framexbottommargin=3pt,
	framexleftmargin=0.2cm,
	framesep=0pt,
	rulesep=2.0pt,
	backgroundcolor=\color{gray95},
	rulesepcolor=\color{black},
	%
	stringstyle=\ttfamily,
	showstringspaces = false,
	basicstyle=\small\ttfamily,
	commentstyle=\color{gray45},
	keywordstyle=\bfseries,
	%
	numbers=left,
	numbersep=15pt,
	numberstyle=\tiny,
	numberfirstline = false,
	breaklines=true,
}

% Minimizar fragmentado de listados
\lstnewenvironment{listing}[1][]
{\lstset{#1}\pagebreak[0]}{\pagebreak[0]}

% Estilo de codigo para la consola
\lstdefinestyle{consola}{
	language=bash,
	breaklines=true,
	basicstyle=\footnotesize\bf\ttfamily,
	backgroundcolor=\color{gray85},
	rulesepcolor=\color{myturquoise},
	numbers=none,
}

% -- FIN ESTILO PARA EL CODIGO ---------------------------------------