\chapter{Conclusiones}
A lo largo de este documento se ha introducido al uso tanto por separado como en conjunción de las herramientas Docker y ROS. Se ha buscado darle un enfoque para hacer uso de esas herramientas en sistemas embebidos con aplicaciones robóticas. De todas maneras no ha sido ni es objetivo de este documento implementar sistemas concretos, sino servir como referencia y como guía para implementarlos.

El documento es el fruto de la investigación de ambas herramientas, apoyado con pruebas, ejemplos y explicaciones claras y comprensibles para cualquiera con conocimientos informáticos que no conozca dichas herramientas. Contiene un cuidado formato y contiene un gran número de referencias para respaldar y ampliar los datos y conceptos del documento.

	\section{Sobre Docker y ROS}
	Sobre Docker decir que, en mi opinión es el futuro de la virtualización. A diferencia de herramientas de virtualización tradicionales como VirtualBox o VMware, Docker ofrece una optimización de recursos, una simplicidad en la administración de sistemas y una escalabilidad que lo hace mucho mejor a sus competidores. Aunque quizás Docker todavía esté en un estado de desarrollo en el que aplicarlo a producción no sea todavía la mejor opción, a corto plazo pasara a ser una solución que cambiará por completo la administración de sistemas.
	
	Sobre ROS decir que para aplicaciones robóticas es una gran solución, y mas teniendo en cuenta la proliferación en estos últimos años de microcomputadores como la Raspberry Pi, en lo que se puede por poco dinero obtener un entorno para ROS y usarlo para aplicaciones robóticas con un bajo coste. Además, teniendo en cuenta la gran cantidad de librerías y de documentación que hay al respecto, hace más sencillo meterse en un mundo tan técnico como el de la robótica.
	
	\section{Sobre la asignatura}
	Todo el trabajo de este documento está englobado en la asignatura de Administración de Sistemas, a la cual se le ha dado un enfoque práctico mediante un trabajo con una gran carga de investigación y de aprendizaje autónomo como éste.
	
	por una parte trabajar con herramientas tan novedosas como Docker  y poder profundizar en ellas es algo muy interesante, habiendo pasado de ni conocer la herramienta a entender su funcionamiento y comprender el enfoque de una herramienta de virtualización que difiere de las herramientas tradicionales.
	
	Por otra parte, y en mi humilde opinión, puede echarse un poco en falta profundizar en temas como la administración de sistemas Linux tradicional o Shell Scripting, que son temas que me resultan de interés. Aunque comprendo que el tiempo es limitado y que mediante el trabajo se ha trabajado con muchas cosas como virtualización, redes, configuración de entornos, además de haber aprendido a usar una herramienta de documentación tan potente como es \LaTeX.
	
	En definitiva puedo decir que la realización del trabajo me ha parecido una experiencia satisfactoria y con la que he aprendido mucho.