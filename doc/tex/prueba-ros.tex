\chapter{Prueba con nodos ROS}
Asumiendo que ya tenemos creadas varios nodos de ROS conectados entre si, a los cuales denominaremos como \textbf{ros1}, \textbf{ros2} y \textbf{ros3}, vamos a crear un pequeño ejemplo de este tipo para comprender mejor su funcionamiento. En el ejemplo, El publisher publicará una serie de datos, que será recibidos por el subscriber una vez se haya suscrito al topic en el que se estén publicando. Para hacerlo lo más simple posible, los datos que se pasarán serán una serie de números enteros.

	\section{Código de la prueba}
	
	\subsection{Publisher}
	\lstinputlisting[style=cpp,numbers=left]{../code/pub-sub-example/talker.cpp}
	
	\subsection{Subscriber}
	\lstinputlisting[style=cpp,numbers=left]{../code/pub-sub-example/listener.cpp}
	
	\subsection{CMakeLists}
	\lstinputlisting[style=makefile,numbers=left]{../code/pub-sub-example/CMakeLists.txt}
	
	\section{Construir el paquete}
	
	\section{Ejecución de la aplicación}